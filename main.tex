\documentclass{article}
\usepackage{booktabs}
\usepackage{array}
\usepackage{wrapfig}
\usepackage{multirow}
\usepackage{tabularx}
\usepackage{graphicx}
\usepackage{pgf, tikz}
\usepackage{amsmath}
\usepackage{amssymb}
\usepackage{mdframed}
\usepackage{polyglossia}
\setdefaultlanguage{english}
\usepackage[style=alphabetic,minalphanames=3,maxbibnames=99]{biblatex}
\usepackage{hyperref}
\usepackage{cleveref}
\crefname{section}{\S}{\S\S}
\Crefname{section}{\S}{\S\S}

\addbibresource{references.bib}

\usetikzlibrary{arrows, automata}

\def\bitcoinA{%
  \leavevmode
  \vtop{\offinterlineskip %\bfseries
    \setbox0=\hbox{B}%
    \setbox2=\hbox to\wd0{\hfil\hskip-.03em
    \vrule height .3ex width .15ex\hskip .08em
    \vrule height .3ex width .15ex\hfil}
    \vbox{\copy2\box0}\box2}}

\title{WabiSabi - Draft v0.3}
\author{Ádám Ficsór, Yuval Kogman, István András Seres}
\date{\today}

\begin{document}

\maketitle

\begin{abstract}
  Bitcoin's model for value transfer is a publicly verifiable ledger of transactions, with ownership of coins defined in terms of public keys. Although this design lacks strong guarantees for privacy, it does not rule it out.
  Despite the potential for private use, research has shown that users' activity can often be traced in practice.
  This lack of inherent privacy has given rise to business models built on dragnet surveillance of Bitcoin users which is threat to Bitcoin's fungibility.

  A number of methods have been proposed to mitigate these issues. Among these, CoinJoin is an approach to structuring transactions which adds ambiguity and breaks common assumptions that underlie heuristics used for surveillance.
  Current implementations of CoinJoin transactions present a number of limitations which may partly explain their lack of widespread adoption.

  This work introduces a new protocol for centrally coordinated CoinJoin implementations utilizing keyed verification anonymous credentials and homomorphic value commitments. This is an improvement on the current approach which utilizes blind signatures in both privacy and flexibility, enabling additional use cases and reduced overheads.
\end{abstract}

\section{Introduction}

Bitcoin transactions redistribute funds by consuming outputs of previous transactions as inputs, and creating new outputs. The protocol rules enforced by the network ensure that transactions do not arbitrarily inflate the monetary supply and that outputs are spent at most once. While some newer cryptocurrencies use more sophisticated approaches to defining such rules, in Bitcoin the amounts as well as the specific outputs being spent are broadcast in the clear as part of the transaction. This presents significant challenges for transacting with privacy\footnote{In this work we restrict the discussion of Bitcoin privacy to that of transactions, but there are other considerations especially at the network layer. For a more comprehensive discussion see \url{https://en.bitcoin.it/wiki/Privacy}.} as shown already in some of the earliest academic studies of Bitcoin~\cite{reid2013analysis,ron2013quantitative,androulaki2013evaluating,ober2013structure,moeser2013inquiry,meiklejohn2013fistful}.

The conditions for spending a transaction output are specified in its \texttt{scriptPubKey}, typically requiring that the spending transaction be signed by a specific public key. The signatures authorizing a transaction usually commit to the transaction in its entirety, which makes it possible for mutually distrusting parties to jointly create transactions without risking misallocation of funds: participants will only sign a proposed transaction after confirming that their desired outputs are included.

Chaumian CoinJoin~\cite{mizrahi2013blind,maxwell2013coinjoin,zerolink} is a privacy enhancing technique that uses this atomicity property and Chaumian blind signatures~\cite{chaum1983blind} to construct collaborative Bitcoin transactions, also known as CoinJoins. Participants connect to a server, known as the coordinator, and submit their inputs and outputs using different anonymity network identities. That alone would provide anonymity but since outputs are unconstrained it's not robust against malicious users who can disrupt the protocol by claiming more than their fair share. To mitigate this the coordinator provides blind signatures representing units of standard denominations in response to submitted inputs. Output registration is authorized by presenting a valid signature without the coordinator being able to link the signature to a specific input.

The use of standard denominations in the resulting CoinJoin transaction obscures the relationship between individual inputs and outputs, making the origins of each output ambiguous. Unfortunately standard denominations also present limitations, such as requiring change outputs from these CoinJoin transactions, and limiting the usefulness of the privacy enhanced outputs for payments of arbitrary amounts.

\subsection{Limitations of Wasabi}
In this work, we are aiming to improve on ZeroLink~\cite{zerolink} as implemented by Wasabi, the most popular Chaumian CoinJoin implementation for Bitcoin. We identify several privacy shortcomings and inefficiencies of Wasabi CoinJoins. Some metrics comparing Wasabi, Samourai and other apparent CoinJoin transactions are provided. ``Other'' includes JoinMarket, but also has an inherent false positive error given these transactions are identified heuristically.

\subsubsection{Round denominations}

Due to the nature of blind signatures, mixed outputs of Wasabi CoinJoins are restricted to a fixed set of value denominations which are multiples of a base denomination\footnote{Approximately $0.1$\bitcoinA{}}.

This creates friction when sending or receiving arbitrary amounts of Bitcoin, as the fixed denomination generally creates change which is smaller, both when mixing and when spending mixed outputs.

We define \emph{CoinJoin inefficiency} as the fraction of non-mixed change outputs in a CoinJoin transaction, see \cref{fig:cjinefficiency}.

\begin{figure}[h!]
    \centering
    \includegraphics[scale=0.4]{Figures/CJInefficiency.pdf}
    \caption{CoinJoin inefficiency of various privacy-focused Bitcoin wallets.}
    \label{fig:cjinefficiency}
\end{figure}

\subsubsection{Minimum denomination}

In order to pariticpate a user's combined input amount must be greater or equal to the base denomination.\footnote{The observed base denominations in Wasabi's CoinJoins is usually slightly higher than the announced, agreed upon base denomination. Thus participants sometimes get back slightly more value in the CoinJoins than how much they put into.} We observe, that considerable portion of CoinJoin inputs are less than this minimum denomination, see \cref{fig:minimumdenomination}.

\begin{figure}[h!]
    \centering
    \includegraphics[scale=0.4]{Figures/SmallValueInputsWasabi.pdf}
    \caption{Fraction of inputs with value smaller than the minimum denomination in Wasabi CoinJoin transactions.}
    \label{fig:minimumdenomination}
\end{figure}

Even if users are able to provide several smaller value inputs with total value greater than the minimum denomination, the coordinator knows those inputs belong to the same user. In an ideal mixing protocol the coordinator should not obtain more information than the already available blockchain data by coordinating the CoinJoin transaction. This is especially important in practice as identifying  sub-transactions when many multiple inputs and outputs are provided by the same parties is often computationally expensive with currently known algorithms, even without equal amounts being utilized~\cite{maurer2017anonymous}.

Furthermore if users consolidate coins before the CoinJoin in an additional transaction in order to be able to participate in a CoinJoin, then this link is revealed publicly based on the common input ownership heuristic~\cite{meiklejohn2013fistful}.

\subsubsection{Variable denominations} Since users pay mining and coordination fees the denominations are gradually reduced between rounds in order to make it possible for users to mix several times without providing additional inputs. This introduces a perverse incentive to minimize coordination fees by remixing in quick succession in order, resulting in a smaller anonymity set than with time-staggered remixes.

\subsubsection{Block-space efficiency}

The rigidity of the current transaction structure, i.e. fixed denominations, constrains users' unspent transaction output set structure as well. These limitations force users to consolidate their coins (see \cref{fig:postmixmerging}) and create additional intermediate outputs with constrained amounts when interspersing CoinJoin transactions with transactions that send or receive value.

\begin{figure}[h!]
    \centering
    \includegraphics[scale=0.4]{Figures/postMixInputMerging.pdf}
    \caption[]{Average number of inputs from the first post-mix transactions in various CoinJoin schemes.\footnotemark}
    \label{fig:postmixmerging}
\end{figure}

\footnotetext{The high numbers of consolidation of other CoinJoin like transactions suggest the algorithm used to identify them yielded a large number of false positive results.}

\subsubsection{Lack of privacy-enhanced payments} Currently Wasabi supports neither payments from a CoinJoin, nor payments in a CoinJoin. Payments from a CoinJoin would protect sender privacy and improve efficiency by requiring fewer intermediate outputs. Payments within a CoinJoin would protect both sender and receiver privacy, and since it is a form of PayJoin\footnote{\url{https://en.bitcoin.it/wiki/PayJoin}} would also improve privacy since it introduces degrees of freedom in the interpretation of CoinJoins.

\subsection{Our Contribution}

We introduce WabiSabi, a generalization of Chaumian CoinJoin based on a Keyed-Verification Anonymous Credentials-based (KVAC) scheme~\cite{chase2019signal}. The use of KVACs replaces blind signatures' standard denominations with homomorphic amount commitments, similar to Confidential Transactions~\cite{maxwell2016confidential}, but retains the invariant that the sum of any one participant's outputs does not exceed that of their inputs without the coordinator learning the underlying values. In addition to being more flexible this improves privacy compared to standard denominations, since smaller inputs can be combined and change outputs created with the same unlinkability guarantees as the privacy enhanced outputs\footnote{Note that the cleartext amounts appearing in the final transaction might still link individual inputs and outputs.}.

WabiSabi can be instantiated to construct a variety of CoinJoin transaction structures that depart from the standard output denomination convention, such as SharedCoin\footnote{\url{https://github.com/sharedcoin/Sharedcoin}} and CashFusion\footnote{\url{https://github.com/cashshuffle/spec}} style transactions and Knapsack~\cite{maurer2017anonymous} mixing. Additionally it makes possible arbitrary consolidation of outputs, minimizing unmixed change, relaxing minimum required denominations, improved block space efficiency, making payments from CoinJoins, as well as payments within CoinJoins, also known as PayJoins.

\section{Preliminaries}

Hereby we give an informal and high-level description of applied cryptographic primitives. In the following the security parameter is denoted as $\lambda$.

\subsection{Commitment schemes}
A commitment scheme allows a party to commit to a message without enabling them to change their mind about the committed message after publishing the commitment. On the other hand the commitment should not reveal anything about the committed message.

\noindent$\mathsf{Commit}(m,r)\xrightarrow{}\mathcal{C}$. The $\mathsf{Commit}$ algorithm generates a commitment $\mathcal{C}$ to message $m$ using randomness $r$.

\noindent$\mathsf{OpenCom}(\mathcal{C},m,r)\xrightarrow{}\{\mathit{True},\mathit{False}\}$: one can verify the correctness of the opening of a commitment by checking $\mathcal{C}\stackrel{?}{=}\mathsf{Commit}(m,r)$. If equality holds the algorithm outputs $\mathit{True}$, otherwise $\mathit{False}$.

For ease of understanding the reader can assume in the following that the commitment scheme is instantiated as a Pedersen commitment.

\subsection{MAC}
A message authentication code (MAC) ensures the integrity of a message and consists of the following three probabilistic polynomial-time algorithms.

\noindent$\mathsf{GenMACKey}(\lambda)\xrightarrow{}{\mathsf{sk}}$. a party generates a secret key $\mathsf{sk}$ for MAC generation and verification.

\noindent$\mathsf{MAC}_{\mathsf{sk}}(m)\xrightarrow{}t$. one can generate a MAC $t$ on a message a $m$ by using their $\mathsf{sk}$.

\noindent$\mathsf{VerifyMAC}_{\mathsf{sk}}(m,t)\xrightarrow{}\{\mathit{True},\mathit{False}\}$. The issuer of the MAC can verify a MAC $t$ given the message $m$ it was issued on.

The reader might intuitively think of a MAC as the symmetric-key counterpart of digital signatures. They both have the same goals and similar security requirements, however a MAC is not publicly verifiable.

\subsection{Zero-knowledge proofs of knowledge}
A very high-level, and hence somewhat imprecise, description of zero-knowledge proofs is provided. This protocol involves a prover and a verifier. A prover wishes to prove that a relation $\mathcal{R}$ holds with respect to a secret input $w$, called witness, and public input $x$. Specifically, the prover wants to prove that $(x, w) \in \mathcal{R}$ without revealing anything about $w$.

\noindent$\mathsf{Prove}(x,w,\mathcal{R})\xrightarrow{}\pi$. Given $x$ and the private witness $w$ the prover generates a proof $\pi$.

\noindent$\mathsf{Verify}(x,\pi,\mathcal{R})\xrightarrow{}\{\mathit{True},\mathit{False}\}$. The verifier is given the proof $\pi$ and $x$ and decides whether the prover knows a secret $w$ such that $\mathcal{R}(x,w)=1$ holds.

\section{Protocol Overview}

\subsection{Phases}

A CoinJoin round consists of an Input Registration, an Output Registration and a Transaction Signing phases. To defend against Denial of Service attacks it is important to ensure the inputs of users who do not comply with the protocol are identified, thus these inputs can be excluded from the following rounds in order to ensure completion of the protocol.

\begin{enumerate}
    \item While identifying non-compliant inputs during Input Registration phase is trivial, there is no reason for issuing penalties at this point.
    \item Identifying non-compliant inputs during Output Registration phase is not possible, thus this phase always completes and progresses to the Signing Phase.
    \item During Signing Phase, inputs those do not sign are non-compliant inputs and they shall be issued a penalties.
\end{enumerate}

The cryptography in WabiSabi ensures honest participants always agree to sign the final CoinJoin transaction from the coordinator assuming it accepts the outputs they request. Anonymous credentials allow the coordinator to verify that amounts of each user's output registrations are funded by input registrations without learning specific relationships between inputs and outputs.

\subsection{Credentials}

The coordinator issues anonymous credentials which authenticate attributes in response to registration requests. We use keyed-verification anonymous credentials (introduced in~\cite{chase2014algebraic}), in particular the scheme from~\cite{chase2019signal} which supports group attributes (attributes whose value is an element of the underlying group $\mathbb{G}$). A user can then prove possession of a credential in zero knowledge in a subsequent registration request, without the coordinator being able to link it to the registration from which it originates.

In order to facilitate construction of a CoinJoin transaction while protecting the privacy of participants we instantiate the scheme with a single group attribute $M_a$ which encodes a confidential Bitcoin amount as a Pedersen commitment. These commitments are never opened, instead properties of the underlying values are proven in zero knowledge, allowing the coordinator to verify validity of the requests made by honest participants. Ideally the coordinator would not learn anything that can't be learned from the resulting CoinJoin transaction.

\subsection{Registration}

For intuition we first describe a pair of protocols where credentials are issued during input registration, and then presented at output registration. $k$ denotes the number of credentials used in registration requests, and $a_{\mathit{max}} = 2^{51}-1$ constrains the amount value ranges\footnote{$\log_2(2099999997690000) \approx 50.9$}. To improve privacy and efficiency these two cases are then generalized into a unified protocol used for both input and output registration, where every registration involves both presentation and issuance of credentials. This protocol is described in detail in \cref{details}.

In order to maintain privacy clients must isolate registration requests using unique network identities. A single network identity must not expose more than one input or output, or more than one set of requested or presented credentials.

For fault tolerance request handling should be idempotent, allowing requests whose response was lost to be retried without modification, using the same or a new network identity.

\subsubsection{Input Registration}

\begin{figure}[h!]
    \begin{mdframed}
    \begin{enumerate}
        \item The user sends $k$ credential requests with accompanying range and sum proofs to the coordinator:  $((M_{a_i},\pi^{\textit{range}}_{i})^{k}_{i=1},\pi^{sum},a_{\textit{in}})$.
        \item The coordinator verifies the received proofs. If they are not verified it aborts the protocol, otherwise it issues $k$ MACs on the requested attributes $(\mathsf{MAC}_\mathsf{sk}(M_{a_i}), \pi_i^{\mathrm{iparams}})^{k}_{i=1}$.
    \end{enumerate}

\end{mdframed}
    \caption{Input Registration protocol}
    \label{fig:inputreg}
\end{figure}

The user submits an input of amount $a_{\mathit{in}}$ along with $k$ group attributes, $(M_{a_i})$.
She proves in zero knowledge that the sum of the requested sub-amounts is equal to $a_{\mathit{in}}$ and that the individual amounts are positive integers in the allowed range.

The coordinator verifies the proofs, and issues $k$ MACs on the requested attributes, along with a proof of correct generation of the MAC, as in \textit{Credential Issuance} protocol of \cite{chase2019signal}.

\subsubsection{Output Registration}

\begin{figure}[h!]
    \begin{mdframed}
    \begin{enumerate}
        \item The user sends $k$ randomized commitments, a proof of a valid MAC for the corresponding non-randomized commitments, serial numbers with a proof of their validity, and finally a proof of the sum of the amounts: $((C_{a_i},\pi_{i}^{\textit{MAC}},S_i,\pi_i^{\textit{serial}})^{k}_{i=1}, \pi^{\textit{sum}}, a_{\textit{out}})$.
        \item The coordinator verifies proofs and registers requested output iff. all proofs are valid and the serial numbers have not been used before.
    \end{enumerate}
\end{mdframed}
    \caption{Output Registration protocol}
    \label{fig:outputreg}
\end{figure}

To register her output the user randomizes the attributes and generates a proof of knowledge of $k$ valid credentials issued by the coordinator.

Additionally, she proves the serial number is valid. These serial numbers are required for double spending protection, and must be correspond but unlinkable to a specific $M_a$.

Finally, she proves that the sum of her randomized amount attributes $C_a$ matches the requested output amount $a_{\mathit{out}}$, analogously to input registration.\footnote{Note that there is no need for range proofs, since amounts have been previously validated.}

She submits these proofs, the randomized attributes, and the serial numbers. The coordinator verifies the proofs, and if it accepts the output will be included in the transaction.

\subsubsection{Unified Registration}\label{unified}

In order to increase flexibility in a dynamic setting, where a user may not yet know her desired output allocations during input registration, and to allow setting a small\footnote{Specifically, $2 \le k \le 10 \approx \log_2\left(\frac{\mathtt{MAX\_STANDARD\_TX\_WEIGHT} - 58}{274 + 124}\right)$ the maximum number of participants, because although $k=1$ suffices for flexibility it limits parallelism, leaking privacy by temporal fingerprinting. The limit on participant count is because 274 and 124 are the minimum weight units required for a participant with only a single input and output, and 58 is the shared per transaction overhead.} value of $k$ as a protocol level constant to reduce privacy leaks, we can generalize input and output registration into a single unified protocol for use in both phases, which also supports reissuance. For complete definitions see \cref{details}.

\begin{figure}[h!]
    \begin{mdframed}
    \begin{enumerate}
        \item During both input and output registration phases the user submits:
        \begin{itemize}
            \item $k$ credential requests with accompanying range and sum proofs to the coordinator:  $(M_{a_i},\pi^{\textit{range}}_{i})^{k}_{i=1}$
            \item $k$ randomized commitments, proofs of valid credentials issued for the corresponding non-randomized commitments, serial numbers, and proofs of their validity: $(C_{a_i},\pi_{i}^{\mathit{MAC}},S_i,\pi_i^{\textit{serial}})^{k}_{i=1}$
            \item A balance $\Delta_{a}$ and a proof of its correctness $\pi^{\textit{sum}}$
            \item If $\Delta_{a} \ne 0$, an input or output with value $|\Delta_{a}|$.
        \end{itemize}
        \item The coordinator verifies the received proofs, and that the serial numbers have not been used before, and depending on the current phase, $\Delta_{a} \geq 0$ (input) or $\Delta_{a} \leq 0$ (output). If it accepts, it issues $k$ MACs on the requested attributes $(\mathsf{MAC}_\mathsf{sk}(M_{a_i}), \pi_i^{\mathrm{iparams}})^{k}_{i=1}$, and if $\Delta_{a} \ne 0$, registers the input or output with value $|\Delta_{a}|$.
    \end{enumerate}
    \end{mdframed}
    \caption{Unified Registration protocol}
    \label{fig:reissue}
\end{figure}

The user submits $k$ valid credentials and $k$ credential requests, where the sums of the underlying amount commitments must be balanced (\cref{fig:reissue}).

\begin{figure}[h!]
  \begin{mdframed}
    \begin{enumerate}
    \item During input registration phase the user submits $k$ credential requests:  $(M_{a_i},\pi^{\mathit{null}}_{i})^{k}_{i=1}$
    \item The coordinator verifies the received proofs. If it accepts, it issues $k$ MACs on the requested attributes $(\mathsf{MAC}_\mathsf{sk}(M_{a_i}), \pi_i^{\mathrm{iparams}})^{k}_{i=1}$.
    \end{enumerate}
  \end{mdframed}
  \caption{Credential bootstrapping protocol}
  \label{fig:bootstrap}
\end{figure}

To prevent the coordinator from being able to distinguish between initial vs. subsequent input registration requests (which may merge amounts) registration operations credential presentation should be mandatory. Initial credentials can be obtained with an auxiliary bootstrapping operation (\cref{fig:bootstrap}).

\subsection{Signing phase}

The user fetches the finalized but unsigned transaction from the coordinator, and if she sees her registered outputs she will sign her inputs and submit the each signatures separately using the network identity used for the that input's registration.

\subsection{Examples}

To illustrate the above protocols, \cref{fig:ex1,fig:ex2} show how a user might register inputs and outputs when credentials are only presented during the output registration phase and \cref{fig:ex3,fig:ex4} show the unified protocol, when credentials are both presented and requested in every registration request.

Registrations requests are depicted as vertices labeled with $\Delta_a$, a double stroke denoting output registrations. A credential is an edge from the registration in which it was requested to the registration where it was presented, also labeled with the amount. The sum of a vertex's label and the labels of its incoming edges must equal to the sum of the labels of its outgoing edges. Note that edges and their labels are only known to the owners of the credentials. For simplicity we omit credentials with zero value.

\begin{figure}[h!]
  \centering
  \begin{tikzpicture}[
    auto,
    node distance = 1cm,
    shorten  >=1pt,
    ]

    \tikzstyle{every state}=[
    draw = black,
    thick,
    fill = white,
    ]

    \node[state] (i) {10};
    \node[accepting,state] (o1) [above right of=i, xshift=1.75cm] {7};
    \node[accepting,state] (o2) [below right of=i, xshift=1.75cm] {3};

    \path[->] (i) edge node {\tiny{7}} (o1);
    \path[->] (i) edge node {\tiny{3}} (o2);
  \end{tikzpicture}
  \caption{Alice wants to spend an input of amount 10 and create two outputs with amounts 7 and 3 (e.g. a payment and change)}
  \label{fig:ex1}
\end{figure}

\begin{figure}[h!]
  \centering
  \begin{tikzpicture}[
    auto,
    node distance = 1.5cm,
    shorten  >=1pt,
    semithick,
    ]

    \tikzstyle{every state}=[
    draw = black,
    thick,
    fill = white,
    ]

    \node[state] (i1) {6};
    \node[state] (i2) [below of=i1] {4};
    \node[accepting,state] (o1) [right of=i1, xshift=1.75cm] {7};
    \node[accepting,state] (o2) [right of=i2, xshift=1.75cm] {3};

    \path[->] (i1) edge node {\tiny{6}} (o1);
    \path[->] (i2) edge node {\tiny{1}} (o1);
    \path[->] (i2) edge node {\tiny{3}} (o2);
  \end{tikzpicture}
  \caption{Alice wants to combine her inputs valued 6 and 4 and register two outputs as in the previous example.}
  \label{fig:ex2}
\end{figure}

\begin{figure}[h!]
  \centering
  \begin{tikzpicture}[
    auto,
    node distance = 1.75cm,
    shorten  >=1pt,
    semithick,
    ]

    \tikzstyle{every state}=[
    draw = black,
    thick,
    fill = white,
    ]

    \node[state] (i1) {6};
    \node[state] (i2) [right of=i1] {4};
    \node[accepting,state] (o1) [above right of=i2, xshift=1.75cm] {7};
    \node[accepting,state] (o2) [below right of=i2, xshift=1.75cm] {3};

    \path[->] (i1) edge node {\tiny{6}} (i2);
    \path[->] (i2) edge node {\tiny{7}} (o1);
    \path[->] (i2) edge node {\tiny{3}} (o2);
  \end{tikzpicture}
  \caption{Alice wants to spend two inputs and register two outputs using the unified protocol, which allows her to present the credential from her first input registration when registering her second input to combine the amounts.}
  \label{fig:ex3}
\end{figure}

\begin{figure}[h!]
  \centering
  \begin{tikzpicture}[
    auto,
    node distance = 1.75cm,
    shorten  >=1pt,
    semithick,
    ]

    \tikzstyle{every state}=[
    draw = black,
    thick,
    fill = white,
    ]

    \node[state] (i1) {$6_A$};
    \node[state] (i2) [right of=i1] {$4_A$};
    \node[state] (i3) [below right of=i2] {$4_B$};
    \node[accepting,state] (o1) [above right of=i2, xshift=2.5cm] {$3_A$};
    \node[accepting,state] (o2) [below of=o1] {$10_B$};
    \node[accepting,state] (o3) [below of=o2] {$1_B$};

    \path[->] (i1) edge node {\tiny{6}} (i2);
    \path[->] (i2) edge node {\tiny{7}} (i3);
    \path[->] (i2) edge node {\tiny{3}} (o1);
    \path[->] (i3) edge node {\tiny{10}} (o2);
    \path[->] (i3) edge node {\tiny{1}} (o3);
  \end{tikzpicture}
  \caption{Alice wants to pay Bob, who is also participating in protocol. Alice combines her amounts at input registration and reveals a credential corresponding to the payment amount to Bob. Bob presents this credential in his own input registration. Alice registers a change output, and Bob registers two outputs. Only Alice knows the details of the request in which the credential labeled 7 was issued and only Bob knows where it was presented, but both know the amount.}
  \label{fig:ex4}
\end{figure}

\section{Cryptographic Details}\label{details}

Following \cite{chase2019signal}, the credential scheme for the protocol in \cref{unified} is defined over a group \(\mathbb{G}\) of prime order \(q,\) written in multiplicative notation.
$\mathsf{HashTo\mathbb{G}} : \{0,1\}^{*} \mapsto \mathbb{G}$ is a function from strings to group elements, based on a cryptographic hash function\cite{fouque2012indifferentiable}.

We require the following fixed set of group elements for use as generators with different purposes:
\[
\underbrace{G_{w}, G_{w^{\prime}}, G_{x_{0}}, G_{x_{1}}, G_{V}}_{\mathsf{MAC} \text{~and~} \mathsf{Show}}
\qquad
\underbrace{G_a}_{\text{attributes}}
\qquad
\underbrace{G_g, G_h}_{\text{commitments}}
\qquad
\underbrace{G_s}_{\text{serial number}}
\]
chosen so that nobody knows the discrete logarithms between any pair of them, e.g. $G_h = \mathsf{HashTo\mathbb{G}}(``\texttt{h}")$.

This notation deviates slightly from \cite{chase2019signal}, in that we subscript the attribute generators $G_{y_i}$ as $G_a$ instead of using numerical indices, and we require two additional generators $G_g$ and $G_h$ for constructing the attribute $M_a$ as a Pedersen commitments.

As with the generator names, we modify the names of the attribute related components of the secret key
$\mathrm{sk} = (w, w^{\prime}, x_{0}, x_{1}, y_{a}) \in_R {\mathbb{Z}_q}^5$
according to our fixed set of group attributes.

The coordinator parameters
$\mathit{iparams} =  (C_{W}, I)$
are computed as:
\[
C_{W}={G_w}^{w} {G_{w^\prime}}^{w^\prime}
\quad
I=\frac{G_{V}}{{G_{x_0}}^{x_0} {G_{x_1}}^{x_1} {G_a}^{y_a} }
\]
and published as part of the round metadata and are used by the coordinator to prove correctness of issued MACs, and by the users to prove knowledge of a valid MAC.

\subsection{Credential Requests}

For each $i \in [1, k]$ the user chooses an amount $0 \leq a_i < a_{\mathit{max}}$ subject to the constraints of the balance proof (\cref{balance}).

She commits to the amount with randomness $r_i \in_R \mathbb{Z}_q$, and this commitments is the attribute of the requested credentials.:
\[ M_{a_i}={G_h}^{r_i}{G_g}^{a_i} \]

For each amount $a_i$ she also computes a range proof which ensures there are no negative values:
\[
\pi^{\mathit{range}}_i = \operatorname{PK}\left\{\left(a_i, r_i \right) :
M_{a_i} = {G_h}^{r_i}{G_g}^{a_i}
\land
0 \leq a_i < a_{\mathit{max}} \right\}
\]

In credential bootstrap requests the range proofs can be replaced with simpler proofs of $a_i = 0$:
\[
  \pi^{\mathit{null}}_i = \operatorname{PK}\left\{ \left( r_i\right) :
    M_{a_i} = {G_{g}}^{r_i}
  \right\}
\]

We note that if Bulletproofs~\cite{bunz2018bulletproofs} are utilized for the range proofs $\pi^{\textit{range}}_i$ a combined proof will significantly decrease the communication overhead and that some implementations perform the $\pi^{\mathrm{null}}$ optimization already.

\subsection{Credential Issuance}

If the coordinator accepts the requests (see \cref{presentation,serial,balance}), it registers the input or output if one is provided, and for each $i \in [1,k]$ it issues a credential by responding with
$(t_i, V_i) \in \mathbb{Z}_q \times \mathbb{G}$,
which is the output of
$\mathsf{MAC}_{\mathsf{sk}}(M_{a_i})$,
where:
\[
  t_i \in_{R} \mathbb{Z}_{q}
  \qquad
  U_i = \mathsf{HashTo\mathbb{G}}(t_i)
  \qquad
  V_i={G_w}^{w} {U_i}^{x_{0}+x_{1} t_i}{M_{a_i}}^{y_a}
\]


To rule out tagging of individual users the coordinator must prove knowledge of the secret key, and that $(t_i, U_i, V_i)$ are correct relative to $\mathit{iparams}=(C_{W}, I)$:

\begin{align*}
  \pi_{i}^{\mathit{iparams}}=\operatorname{PK}\{ & (w, w^{\prime}, x_{0}, x_{1}, y_a): \\
                                                 &C_{W}={G_{w}}^{w} {G_{w^{\prime}}}^{w^\prime} \land \\
                                                 &I=\frac{G_{V}}{{G_{x_{0}}}^{x_0} {G_{x_1}}^{x_1} {G_a}^{y_a}}  \land \\
                                                 &V_i={G_w}^{w}{U_i}^{x_{0}+x_{1}t_i} {M_{a_i}}^{y_a}
                                                   \}
\end{align*}

\subsection{Credential Presentation}\label{presentation}

The user chooses $k$ unused credentials issued in prior registration requests, i.e. valid MACs $(t_i,V_i)_{i=1}^k$ on attributes $(M_{a_i})_{i=1}^k$ .

For each credential $i \in [1, k]$ she executes a the $\mathsf{Show}$ protocol described in~\cite{chase2019signal}:

\begin{enumerate}

\item She chooses
$z_i \in_{R} \mathbb{Z}_{q}$, and computes
$z_{0_i}=-{t_i} {z_i} (\bmod q)$
and the randomized commitments:
\begin{align*}
C_{a_i}     &= {G_a}^{z_i} M_{a_i} \\
C_{x_{0_i}} &= {G_{x_0}}^{z_i} {U_i} \\
C_{x_{1_i}} &= {G_{x_1}}^{z_i} {U_i}^{t_i} \\
C_{V_i}     &= {G_V}^{z_i} V_i
\end{align*}

\item To prove to the coordinator that a credential is valid she computes a proof:
\begin{align*}
\pi_{i}^{\mathit{MAC}}=\operatorname{PK}\{
& (z_i, z_{0_i},t_i): \\
& Z_i =I^{z_i} \land \\
& C_{x_{1_i}} = {C_{x_{0_i}}}^{t_i} {G_{x_0}}^{z_{0_i}} {G_{x_1}}^{z_i} \}
\end{align*}
which implies the following without allowing the coordinator to link $\pi_{i}^\mathit{MAC}$ to the underlying attributes $(M_{a_i})$:
\[
\mathsf{Verify}((C_{x_{0_i}}, C_{x_{1_i}}, C_{V_i}, C_{a_i}, Z_i), \pi_i^{\mathit{MAC}})
\iff
\mathsf{VerifyMAC}_{\mathsf{sk}}(M_{a_i})
\]

\item She sends $(C_{x_{0_i}}, C_{x_{1_i}}, C_{V_i}, C_{a_i},\pi_i^{\mathit{MAC}})$ and the coordinator computes:
\[
Z_i=\frac{C_{V_i}}{{G_w}^w {C_{x_{0_i}}}^{x_0} {C_{x_{1_i}}}^{x_{1}}
{C_{a_i}}^{y_a}
}
\]
using its secret key (independently of the user's derivation), and verifies $\pi_i^{\mathit{MAC}}$.

\end{enumerate}

\subsection{Double-spending prevention using serial numbers}\label{serial}

The user proves that the group element $S_i = {G_s}^{r_i}$, which is used as a serial number, was generated correctly with respect to $C_{a_i}$:
\[ \pi_{i}^{\mathit{serial}}=\operatorname{PK}\{ (z_i,a_i,r_i): S_i = {G_s}^{r_i} \land C_{a_i} = {G_a}^{z_i}{G_h}^{r_i}{G_g}^{a_i} \} \]

The coordinator verifies $\pi_{i}^{\mathit{serial}}$ and checks that the $S_i$ has not been used before (allowing for idempotent registration).

Note that since the logical conjunction of $\pi_i^{\mathit{serial}}$ and $\pi_i^{\mathit{MAC}}$ is required for each credential, and because these proofs share both public and private inputs it is appropriate to use a single proof for both statements.

\subsection{Over-spending prevention by balance proof}\label{balance}

The user needs to convince the coordinator that the total amounts redeemed and the requested differ by the public input $\Delta_{a}$, which she can prove by including the following proof of knowledge:
\[ \pi^{\mathit{sum}} = \operatorname{PK}(\{ (z, \Delta_r) : B = {G_a}^{z} {G_h}^{\Delta_r} \})
\]
where
\[
B = {G_g}^{\Delta_a} \prod_{i=1}^k \frac{C_{a_i}}{M^{\prime}_{a_i}}
\qquad
z = \sum_{i=1}^k z_i
\qquad
\Delta_r = \sum_{i=1}^k r_i - r^{\prime}_i
\]
with $r^{\prime}_i$ denoting the randomness terms in the $(M^{\prime}_{a_i})_{i=1}^k$ attributes of the credentials being requested and $z_i, r_i$ denoting the ones in the randomized attributes $(C_{a_i})_{i=1}^k$ of the credentials being presented.

During the input registration phase $\Delta_{a}$ may be positive, in which case an input of amount $a_{\mathit{in}} = \Delta_{a}$ must be registered with proof of ownership. During the output registration phase $\Delta_{a}$ may be negative, in which case an output of amount $a_{\mathit{out}} = -\Delta_{a}$ is registered. If $\Delta_{a} = 0$ credentials are simply reissued, with no input or output registration occurring.

\subsection{Perfect Hiding}

Note that $S_i$ is not perfect hiding because there is exactly one $r_i \in \mathbb{Z}_q$ such that $S_i = {G_s}^{r_i}$. Similarly, randomization by $z_i$ only protects unlinkability of issuance and presentation against a computationally bounded adversary. Null credentials have the same issue, since the the amount exponent is known to be zero.

To unconditionally preserve user privacy in the event that the hardness assumption of the discrete logarithm problem in $\mathbb{G}$ is broken we can add an additional randomness term $r_i^{\prime}$ used with an additional generator $G_h^{\prime}$ to the amount commitments $M_{a_i}$, and similarly another randomness term $z_i^{\prime}$ and generators $G_a^{\prime}, G_{x_0}^{\prime}, G_{x_1}^{\prime}, G_V^{\prime}$ in order to obtain unconditional unlinkability for the commitments.\footnote{Assuming the coordinator is not able to attack network level privacy and the proofs of knowledge are unconditionally hiding.}

\printbibliography

\end{document}
